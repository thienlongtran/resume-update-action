%-------------------------
% Resume in Latex
% Author : Jake Gutierrez
% Based off of: https://github.com/sb2nov/resume
% License : MIT
%------------------------

\documentclass[letterpaper,11pt]{article}

\usepackage{latexsym}
\usepackage[empty]{fullpage}
\usepackage{titlesec}
\usepackage{marvosym}
\usepackage[usenames,dvipsnames]{color}
\usepackage{verbatim}
\usepackage{enumitem}
\usepackage[hidelinks]{hyperref}
\usepackage{fancyhdr}
\usepackage[english]{babel}
\usepackage{tabularx}
\usepackage{xcolor}
\input{glyphtounicode}


%----------FONT OPTIONS----------
% sans-serif
% \usepackage[sfdefault]{FiraSans}
% \usepackage[sfdefault]{roboto}
% \usepackage[sfdefault]{noto-sans}
% \usepackage[default]{sourcesanspro}

% serif
% \usepackage{CormorantGaramond}
% \usepackage{charter}

\usepackage{fontawesome}


\pagestyle{fancy}
\fancyhf{} % clear all header and footer fields
\fancyfoot{}
\renewcommand{\headrulewidth}{0pt}
\renewcommand{\footrulewidth}{0pt}

% Adjust margins
\addtolength{\oddsidemargin}{-0.5in}
\addtolength{\evensidemargin}{-0.5in}
\addtolength{\textwidth}{1in}
\addtolength{\topmargin}{-.5in}
\addtolength{\textheight}{1.0in}

\urlstyle{same}

\raggedbottom
\raggedright
\setlength{\tabcolsep}{0in}

% Sections formatting
\titleformat{\section}{
  \vspace{-4pt}\scshape\raggedright\large
}{}{0em}{}[\color{black}\titlerule \vspace{-5pt}]

% Ensure that generate pdf is machine readable/ATS parsable
\pdfgentounicode=1

%-------------------------
% Custom commands
\newcommand{\resumeItem}[1]{
  \item\small{
    {#1 \vspace{-2pt}}
  }
}

\newcommand{\resumeSubheading}[4]{
  \vspace{-2pt}\item
    \begin{tabular*}{0.97\textwidth}[t]{l@{\extracolsep{\fill}}r}
      \textbf{#1} & #2 \\
      \textit{\small#3} & \textit{\small #4} \\
    \end{tabular*}\vspace{-7pt}
}

\newcommand{\resumeSubSubheading}[2]{
    \item
    \begin{tabular*}{0.97\textwidth}{l@{\extracolsep{\fill}}r}
      \textit{\small#1} & \textit{\small #2} \\
    \end{tabular*}\vspace{-7pt}
}

\newcommand{\resumeProjectHeading}[2]{
    \item
    \begin{tabular*}{0.97\textwidth}{l@{\extracolsep{\fill}}r}
      \small#1 & #2 \\
    \end{tabular*}\vspace{-7pt}
}

\newcommand{\resumeSubItem}[1]{\resumeItem{#1}\vspace{-4pt}}

\renewcommand\labelitemii{$\vcenter{\hbox{\tiny$\bullet$}}$}

\newcommand{\resumeSubHeadingListStart}{\begin{itemize}[leftmargin=0.15in, label={}]}
\newcommand{\resumeSubHeadingListEnd}{\end{itemize}}
\newcommand{\resumeItemListStart}{\begin{itemize}}
\newcommand{\resumeItemListEnd}{\end{itemize}\vspace{-5pt}}

%-------------------------------------------
%%%%%%  RESUME STARTS HERE  %%%%%%%%%%%%%%%%%%%%%%%%%%%%


\begin{document}

%----------HEADER SETINGS----------
\def \linkedinicon {\faLinkedin}
\def \linkedinlink {https://www.linkedin.com/in/thienlongtran}
\def \linkedintext {/thienlongtran}

\def \emailicon {\faEnvelope}
\def \emaillink {mailto:ttran384@gatech.edu}
\def \emailtext {ttran384@gatech.edu}

\def \githubicon {\faGithub}
\def \githublink {https://github.com/thienlongtran}
\def \githubtext {/thienlongtran}

\def \websiteicon {\faGlobe}
\def \websitelink {https://www.thienlongtran.com/}
\def \websitetext {/thienlongtran.com}

\def \phoneicon {\faPhone}
\def \phonelink {tel:346-204-9381}
\def \phonetext {(346)-204-9381}

\definecolor{LinkColor}{HTML}{3e70bb}

\def \linkedin {\linkedinicon\hspace{3pt}\textcolor{LinkColor}{\href{\linkedinlink}{\linkedintext}}}
\def \email {\emailicon\hspace{3pt}\textcolor{LinkColor}{\href{\emaillink}{\emailtext}}}

\def \github {\githubicon \hspace{3pt}\textcolor{LinkColor}{\href{\githublink}{\githubtext}}}

\def \website {\websiteicon\hspace{3pt}\textcolor{LinkColor}{\href{\websitelink}{\websitetext}}}

\def \phone {\phoneicon\hspace{3pt}\textcolor{LinkColor}{\href{\phonelink}{\phonetext}}}


%----------HEADING----------
% \begin{tabular*}{\textwidth}{l@{\extracolsep{\fill}}r}
%   \textbf{\href{http://sourabhbajaj.com/}{\Large Sourabh Bajaj}} & Email : \href{mailto:sourabh@sourabhbajaj.com}{sourabh@sourabhbajaj.com}\\
%   \href{http://sourabhbajaj.com/}{http://www.sourabhbajaj.com} & Mobile : +1-123-456-7890 \\
% \end{tabular*}

\begin{center}
    \textbf{\Huge \scshape Thien Tran} \\ \vspace{1pt}
    {\email} $|$ 
    {\phone} $|$ 
    {\website} $|$
    {\linkedin} $|$
    {\github}
\end{center}


%-----------EDUCATION-----------
\section{Education}
  \resumeSubHeadingListStart
    %\resumeSubheading
      %{Georgia Institute of Technology}{January 2022 - December 2023 (Expected)}
      %{Master of Science in Computer Science - GPA: 4.00/4.00}{Atlanta, GA}
      %\resumeItemListStart
      %  \resumeItem{\textbf{Coursework:} Knowledge-Based AI, Machine Learning for Trading}
        
      %\resumeItemListEnd
    \resumeSubheading
      {University of New Orleans}{August 2019 - December 2021}
      {Bachelor of Science in Computer Science}{GPA: 3.99/4.00}
      %\resumeItemListStart
        %\resumeItem{\textbf{Leadership:} Google Developers Student Club (DSC Lead), Toastmasters (President), SGA (Senator)}
        %\resumeItem{\textbf{Honors: }President's List (All Semesters), Achievement in Computer Science Scholarship (2020-2021)}
        %\resumeItem{\textbf{Coursework:} Data Structures \& Algorithms, Algorithm Analysis, Python for Data Science \& Artificial Intelligence, Database Management Systems, Cloud Computing, Computer Networks, Operating Systems}
      %\resumeItemListEnd
  \resumeSubHeadingListEnd
  

%
%-----------PROGRAMMING SKILLS-----------
\section{Skills}
 \begin{itemize}[leftmargin=0.15in, label={}]
    \small{\item{
     \textbf{Languages}\hspace*{1cm}{Python, Java, Go, HTML, CSS, JavaScript, SQL} \\
     \textbf{Technologies}\hspace*{0.6cm}{Git, REST APIs, Unity, Jupyter Notebook} \\
     \textbf{Libraries}\hspace*{1.25cm}{NumPy, Pandas, Matplotlib, Scikit-learn} \\
     \textbf{DevOps}\hspace*{1.4cm}{Amazon Web Services (AWS), Terraform, GitHub Actions, Docker, Kubernetes} \\
     \textbf{Certifications}\hspace*{0.4cm}{AWS Solutions Architect - Associate, AWS Cloud Practitioner} 
    }}
 \end{itemize}

%-----------EXPERIENCE-----------
\section{Experience}
  \resumeSubHeadingListStart
     \resumeSubheading
      {Venmo}{May 2022 - Present}
      {Software Engineer}{Austin, TX}
      \resumeItemListStart
      %\resumeItem {Provided  .} TALK ABOUT OBSERVABILITY SERVICE HERE
      %\resumeItem {Provided  .} TALK ABOUT GITHUB ACTIONS LOGS HERE
      \resumeItem {Saved company \$1,142,700 annually in compute and data transfer costs by identifying an unoptimized, continuously running metrics collection process and decreasing the average run-time per job by 96.4\%.}
      \resumeItem {Led a project to improve GitHub Actions observability by designing and developing a proprietary, scalable, high performance microservice using Python, Flask, Docker, AWS Elastic Kubernetes Service, AWS DynamoDB, and DataDog which delivers 14 key real-time metrics about active workflow jobs across Venmo's 1,200+ repositories.}
      \resumeItem {Assumed responsibilities of Venmo's Site Reliability Engineering team including networking, cluster maintenance, operation efficiency and stability, and incident response management.}
      \resumeItem {Educated dozens of Platform Infrastructure and Site Reliability Engineers to support GitHub Actions during on-call rotations by creating an introductory course to Venmo's CI/CD infrastructure and observability.}
      \resumeItem {Established visibility into cost areas and optimization possibilities for \$14,400,000 worth of enterprise CI/CD jobs annually by integrating GitHub Actions Observability API with DataDog cost metrics using Python.}
      \resumeItem {Enabled 24x7x365 reliability of GitHub Actions synthetic tests, recovering up to 25 failures daily, by developing a failure recovery script using Python and deploying it to AWS Lambda using Terraform.}
      \resumeItem {Fostered knowledge sharing, collaboration, and improved platform infrastructure availability and reliability by joining on-call and platform engineering support rotations.}
    \resumeItemListEnd
%     \resumeSubheading
%      {Venmo}{May 2022 - August 2022}
%      {Software Engineer Intern}{Austin, TX}
%      \resumeItemListStart
%      \resumeItem {Improved fault-tolerance of synthetic testing systems of 4 classes of self-hosted GitHub Actions runners by creating a failure recovery script using Python and GitHub APIs that detects and recovers workflow scheduling failures.}
%      \resumeItem {Enabled 24x7x365 reliability of GitHub Actions synthetic tests by deploying failure recovery script to AWS Lambda using Terraform which triggers every 5 minutes, recovers up to 25 failures a day, and only costs \$0.57 a month.}
%      \resumeItem {Led a project to improve GitHub Actions observability by designing and developing a scalable, high performance microservice using Python, Flask, Docker, AWS Elastic Kubernetes Service, and AWS DynamoDB which delivers 11 key real-time metrics about active workflow jobs across Venmo's 1,200+ repositories.}
%    \resumeItemListEnd
%    \resumeSubheading
%      {USAA}{May 2021 - July 2021}
%      {Software Engineer Intern}{Plano, TX}
%    \resumeItemListStart
%      \resumeItem {Reduced cluttering of a qTest archive by 84\% and allowed for easier feature-based auditing by designing a new directory structure for publishing automated infrastructure test results that affected 70 projects.}
%      \resumeItem {Enabled automatic AWS resource tagging on one parameter if not provided by a developer or optional manual tagging otherwise by modifying a custom Terraform provider utilized by 55 projects using GoLang.}
%      \resumeItem {Decreased the cost of conducting network connectivity testing on AWS EC2 instances by 92.38\% by developing a selection of 5 AWS Systems Manager (SSM) testing automations using Bash, Terraform, and GitLab CI/CD, saving the company \$56,700 annually in lost wages and productivity.}
      %\resumeItem {Developed a selection of 5 AWS Systems Manager (SSM) documents and automation using Terraform and GitLab CI/CD which reduced the time it takes to conduct network connectivity testing from 3-4 hours with 2-3 engineers to 35-45 minutes with a single engineer.}
%    \resumeItemListEnd
    
%    \resumeSubheading
%      {University of New Orleans}{January 2021 - May 2021}
%      {Undergraduate Research Assistant}{New Orleans, LA}
%      \resumeItemListStart
%        \resumeItem{Developed immersive eXtended Reality (XR) games using %Unity and C\# under advisement of Dr. Farjana Eishita to discreetly %detect 8 types of cognitive distortions and other mental health %conditions.}
        %\resumeItem{Converted 42 scenes of an existing cognitive distortion %detection game manually from Augmented Reality (AR) to Mixed and %Virtual Reality (MR \& VR) for player-experience (PX) comparisons %between platforms.}
        %\resumeItem{Conducted moderated PX testing on 9 individuals to %identify bugs and ensure effective game-play engagement.}
%      \resumeItemListEnd
      
% -----------Multiple Positions Heading-----------
%    \resumeSubSubheading
%     {Software Engineer I}{Oct 2014 - Sep 2016}
%     \resumeItemListStart
%        \resumeItem{Apache Beam}
%          {Apache Beam is a unified model for defining both batch and streaming data-parallel processing pipelines}
%     \resumeItemListEnd
%    \resumeSubHeadingListEnd
%-------------------------------------------

  \resumeSubHeadingListEnd


%-----------PROJECTS-----------
\section{Projects}
    \resumeSubHeadingListStart
      \resumeProjectHeading
          {\textbf{NBA Totals Investment w/ Machine Learning} $|$ \emph{Python, XGBoost}}{}
          \resumeItemListStart
            \resumeItem{Generated 13.62\% portfolio return per month with a 2.42 Sharpe Ratio by developing an NBA Sports Betting Strategy on Totals Market's daily Over/Under (OU) Lines using a configuration of 2.5\% of portfolio per trade.}
            \resumeItem{Trained an XGBoost model to predict OU results with 54.3\% accuracy, gaining an expected 3.66\% return per bet.}
            %\resumeItem{Automated betting by utilizing CloudBet APIs to get latest data and place bets right before the start of games.}
            \resumeItem{Enabled real-time trade monitoring by integrating DataDog with live scores and FanDuel's OU Line movements.}
          \resumeItemListEnd
      \resumeProjectHeading
          {\textbf{Stocks Simple Moving Average} $|$ \emph{Python, Amazon Web Services}}{}
          \resumeItemListStart
            \resumeItem{Built an AWS pipeline that computes the Simple Moving Average (SMA) of historical OHLC-type stocks.}
            \resumeItem{Created the cloud infrastructure using the AWS Python SDK (Boto3) to automatically initialize and connect two S3 buckets, two Lambda functions, one SNS topic, and one DynamoDB NoSQL database table.}
            \resumeItem{Decreased the time it takes to calculate SMA by 99.87\% compared to manual calculation.}
          \resumeItemListEnd
      \resumeProjectHeading
          {\textbf{Warframe Inventory Market Info} $|$ \emph{Python, OpenCV, PyTesseract}}{}
          \resumeItemListStart
            \resumeItem{Developed a program that automatically gathers 4 different economic attributes about users’ in-game Warframe inventory items, saving users about 52 seconds of work per item page compared to manual calculation.}
            \resumeItem{Generated a list of users' inventory items using OpenCV to isolate item names from the inventory-screen image by thresholding the text colors, and using PyTesseract to read and save the remaining text.}
            \resumeItem{Enabled better investment decisions and comparisons by collecting the average currency price of the 10 current cheapest live web market value sell-orders using the warframe.market API for each item in users' inventory.}
          \resumeItemListEnd
    \resumeSubHeadingListEnd


%-------------------------------------------
\end{document}
